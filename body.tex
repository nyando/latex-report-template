\chapter{Hauptteil}

\section{Anfang des Hauptteils}

Hier kommt der Hauptteil.
In den Hauptteil kommt die eigene Arbeit, die gemacht wurde.
Planung und Implementierung von Softwareprojekten, oder sowas (s. \cite{knuth84}).

\section{Lorem}

\lipsum[1-3]

\subsection{Sublorem}

\begin{lstlisting}[language=c]
    #include <foo.h>

    int main(int argc, char *argv[])
    {
        printf("Hello World!\n");
        return 0;
    }
\end{lstlisting}
\captionof{lstlisting}{Das simpelste Programm, das es gibt.}

\lipsum[2-3]

\begin{table}[h]
    \centering
    \begin{tabular}{c|c}
        $n$ & $f(n)$ \\ \hline
        1   & 0      \\
        2   & 1      \\
        3   & 1      \\
        4   & 2      \\
        5   & 3      \\
        6   & 5      \\
        7   & 8      \\
    \end{tabular}
\end{table}
\captionof{table}{Eine Tabelle mit irgendwelchem Zeugs drin}

\subsubsection{Subsublorem}

Hier noch eine Liste der Abkürzungen und Fachbegriffe in dieser Arbeit:

\begin{itemize}
    \item \gls{report}
    \item \gls{programminglanguage}
    \item \gls{dummy}
\end{itemize}

\begin{enumerate}
    \item \acrshort{rtfm}
    \item \acrshort{gnu}
    \item \acrshort{wtf}
    \item \acrshort{fubar}
\end{enumerate}

\lipsum[2-4]

\subsubsection{Subsubipsum}

\begin{table}[h]
    \centering
    \begin{tabular}{c|ccccccc}
        $n$    & 1 & 2 & 3 & 4 & 5 & 6 & 7 \\ \hline
        $f(n)$ & 0 & 1 & 1 & 2 & 3 & 5 & 8 \\
    \end{tabular}
\end{table}
\captionof{table}{Eine andere Tabelle mit dem gleichen Zeugs drin}

\lipsum[4]

\begin{center}
    \includegraphics[width=\textwidth]{lipsum_logo_1}
    \captionof{figure}{Ein beliebiges Logo}
\end{center}

\lipsum[5]

\subsection{Subipsum}

\lipsum[1]

\begin{center}
    \includegraphics[width=0.5\textwidth]{lipsum_logo_2}
    \captionof{figure}{Ein anderes beliebiges Logo}
\end{center}

\lipsum[2]

\section{Ipsum}

\begin{lstlisting}[language=c]
    #include <stdio.h>

    int main(int argc, char *argv[])
    {
        int a = 0, b = 1;
        
        for (int i = 0; i < 20; i++)
        {
            a = b + a;
            b = a + b;
        }

        printf("%i", b);

        return 0;
    }
\end{lstlisting}
\captionof{lstlisting}{Irgendwas wird hier berechnet}

\lipsum[2-4]